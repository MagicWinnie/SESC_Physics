\documentclass{article}

\usepackage[utf8x]{inputenc}
\usepackage[english,russian]{babel}
\usepackage{cmap}
\usepackage{commath}
\usepackage{amsmath}
\usepackage{amsfonts}
\usepackage{mathtools}
\usepackage{amssymb}
\usepackage{parskip}
\usepackage{titling}
\usepackage{color}
\usepackage{hyperref}
\usepackage{cancel}
\usepackage{enumerate}
\usepackage{multicol}
\usepackage{graphicx}
\usepackage{docmute}
\usepackage{titlesec}
\usepackage[font=small,labelfont=bf]{caption}
\usepackage[a4paper, left=2.5cm, right=1.5cm, top=2.5cm, bottom=2.5cm]{geometry}

\graphicspath{ {./images/} }
\setlength{\droptitle}{-3cm}
\hypersetup{ colorlinks=true, linktoc=all, linkcolor=blue }
\pagenumbering{arabic}

\setcounter{secnumdepth}{4}
\setcounter{tocdepth}{4}
\titleformat{\paragraph}
{\normalfont\normalsize\bfseries}{\theparagraph}{1em}{}
\titlespacing*{\paragraph}
{0pt}{3.25ex plus 1ex minus .2ex}{1.5ex plus .2ex}

\begin{document}

\section{Электричество.}
    \subsection{Электростатика.}
        \subsubsection{Два вида заряда.}
            \paragraph{Элементарный электрический заряд, электроны и протоны.}
                Элементарный заряд есть заряд электрона(отрицательный). \(\abs{e} = \abs{p} \approx 1.6 \cdot 10^{-19} \) Кл.

                \(m_e \approx 9.1 \cdot 10^{-31}\) кг.
                
                \(m_p \approx 1840m_e\).

        \subsubsection{Закон сохранения электрического заряда.}
                Пусть есть замкнутая система в которой каким-то образом изменяется суммарный заряд. Тогда, для двух произвольных сумм зарядов \(q\) и \(Q\)(в момент времени раньше и позже соответсвенно), выполняется:

                \(Q_0 = Q - q\), \(Q_0\) --- \(const\) в замкнутой системе.
        \subsubsection{Взаимодействие зарядов.}
                Разноимённые заряды притягиваются, одноимённые отталкиваются. Заряды взаимодействуют на любом расстоянии, вне зависимости от того, что находится между ними.
            \paragraph{Закон Кулона.}
                \(F = k \frac{q_1 q_2}{l^2}\) --- справедливо только для точечных зарядов, причём геометрический размер тел должен быть много меньше \(l\).

                \(\vec{F} = k \frac{q_1 q_2}{l^2} \cdot \frac{\vec l}{l}\).
        \subsubsection{Кулон как единица электрического заряда в системе СИ.}
                Предположим два одноимённых заряда в \(1\) Кл находятся в \(1\) км друг от друга. Тогда сила их взаимодействия(при \(k \approx 10^{10}\)) \(F = 10^{10} \frac{1 \cdot 1}{10^{3 \cdot 2}} = 10^4\)(Н) --- \(1\) тонна(!!!).
        \subsubsection{Сравнение электрических и гравитационных взаимодействий.}
                Пусть есть два электрона, находящихся в невесомости на расстоянии \(l\). Посчитаем отношение Кулоновских сил к силам гравитационным.

                \(\frac{F_{\textrm{к}}}{F_G} = \frac{\frac{k e^2}{\cancel{l^2}}}{G \frac{m_e^2}{\cancel{l^2}}} = \frac{k e^2}{G m^2} = 5 * 10^{42}\ \Rightarrow\) во взаимодействии точечных зарядов гравитацией можно пренебрегать.
        \subsubsection{Электрическое поле.}
                Электрическое поле --- это векторное поле, действующее вокруг частиц обладающих электрическим зарядом.
            \paragraph{Действие электрического поля на электрические заряды.}
                Главным свойством электрического поля является действие его на электрические заряды с некоторой силой. По этому действию устанавливается факт его существования. Действие поля на единичный заряд — напряженность поля — является одной из его основных ха­рактеристик, по которой изучается распределение поля в пространстве. Действие заряженного тела на окружающие тела проявляется в виде сил притяжения и отталкивания, стремящихся поворачивать и перемещать эти тела по отношению к заряженному телу.
            \paragraph{Напряженность электрического поля.}
                Возьмём заряженное тело с зарядом \(Q\). Сила Кулона для произвольного заряда \(q\): \(F = q \cdot (\frac{k Q}{l^2}) = q \cdot E\). Где \(E = \frac{k Q}{l^2}\) --- напряжённость электрического поля заряда \(Q\).
            \paragraph{Единичный заряд и пробный заряд.}
                Поляризация --- эффект, при котором в нейтрально заряженных телах, при воздействии достаточно сильных зарядов, по разные края тела, так или иначе, собираются разноимённые заряды, тем самым образуя электрическое поле.

                Пробный заряд --- такой заряд, величина которого не создаёт поляризацию тел(т.е. позволяет точно измерить величину напряжённости поля).
            \paragraph{Сила, действующая на заряд в однородном электрическом поле.}
                Для заряда \(q\) в поле \(\vec E\):

                \(\vec F = q \cdot \vec E\)
            \paragraph{Принцип суперпозиции электрических полей.}
                Напряженность электростатического поля, создаваемого в данной точке системой зарядов, есть векторная сумма напряженности полей отдельных зарядов: \(\vec E = \sum{\vec{E_i}}\).
            \paragraph{Силовые линии.}
                Пусть есть два разноимённых заряда \(+q\) и \(-Q\). Рассмотрим силы, действующие на положительный пробный заряд в произвольных точках пространства в близи этих зарядов.
                
                %1_1_6_5-1

                Если расписать все такие силы для каждой точки пространства, получим картину(для наглядности изображены только несколько линий):
                
                %1_1_6_5-2

                Это и есть силовые линии.

                Силовые линии направлены от положительных к отрицательным зарядам. Положительные заряды в поле движутся по направлению силовых линий, отрицательные --- против.
        \subsubsection{Диэлектрики.}
            \paragraph{Диэлектрики в электрическом поле, поляризация.}

            \paragraph{Условия на границе двух диэлектриков.}
            \paragraph{Вектор электрической индукции.}
        \subsubsection{Поток вектора напряженности электрического поля.}
        \subsubsection{Теорема Гаусса.}
        \subsubsection{Напряженность электрического поля в простейших системах.}
            \paragraph{Точечный заряд.}
            \paragraph{Система точечных зарядов.}
            \paragraph{Однородно заряженная плоскость.}
            \paragraph{Однородно заряженная сфера.}
            \paragraph{Однородно заряженный шар.}
            \paragraph{Расчет напряженности электрического поля через теорему Гаусса.}
            \paragraph{Расчет напряженности электрического поля через телесный угол.}
        \subsubsection{Силовое воздействие электрического поля на поверхность (давление).}
        \subsubsection{Потенциальность электростатического поля.}
            \paragraph{Работа по замкнутому контуру в электростатическом поле.}
            \paragraph{Потенциал электрического поля.}
            \paragraph{Разность потенциалов.}
            \paragraph{Потенциал электрического поля в простейших системах.}
                \textbf{Точечный заряд.}
                
                \textbf{Система точечных зарядов.}
                
                \textbf{Заряженное кольцо (на оси).}
                
                \textbf{Однородно заряженная плоскость (разность потенциалов).}
                
                \textbf{Однородно заряженная сфера.}
                
                \textbf{Осциллограф.}
                
            \paragraph{Эквипотенциальные поверхности.}
            \paragraph{Силы изображения.}
            \paragraph{Заземление.}
        \subsubsection{Энергия электрического поля и обусловленная электрическим полем.}
            \paragraph{Энергия точечного заряда в электрическом поле.}
            \paragraph{Плотность энергии электрического поля.}
            \paragraph{Энергия взаимодействия точечных зарядов.}
            \paragraph{Энергия взаимодействия сложных электрических систем.}
            \paragraph{Энергия заряженной сферы.}
        \subsubsection{Проводники в электрическом поле.}
            \paragraph{Экранирование.}
            \paragraph{Метод электростатических изображений.}
            \paragraph{Равновесие зарядов на проводниках и энергия.}
        \subsubsection{Электрическая емкость.}
            \paragraph{Конденсатор.}
            \paragraph{Связь емкости, напряжения и заряда на конденсаторе.}
            \paragraph{Расчет емкости плоского конденсатора.}
            \paragraph{Расчет емкости простых конденсаторов (сфера, цилиндр).}
            \paragraph{Составные емкости.}
                \textbf{Последовательное соединение емкостей.}

                \textbf{Параллельное соединение емкостей.}

                \textbf{Емкостные цепи.}

                \textbf{Симметричные и несимметричные емкостные цепи.}
                
            \paragraph{Реальные системы емкостей.}
            \paragraph{Энергия электрического поля конденсатора.}
        \subsection{Электродинамика.}
        \subsubsection{Постоянный электрический ток.}
            \paragraph{Сила тока.}
            \paragraph{Плотность электрического тока.}
        \subsubsection{Электрическое сопротивление.}
            \paragraph{Удельное сопротивление и удельная электропроводность (проводимость).}
            \paragraph{Выражение сопротивления через удельное сопротивление.}
        \subsubsection{Напряжение.}
            \paragraph{Падение напряжения и разность потенциалов.}
        \subsubsection{Закон Ома.}
        \subsubsection{Амперметр и вольтметр.}
            \paragraph{Идеальный амперметр.}
            \paragraph{Идеальный вольтметр.}
            \paragraph{Реальные амперметр и вольтметр.}
        \subsubsection{Составные сопротивления.}
            \paragraph{Последовательное соединение проводников.}
            \paragraph{Параллельное соединение проводников.}
            \paragraph{Цепи сопротивлений, смешанное соединение проводников.}
            \paragraph{Симметричные и несимметричные цепи сопротивлений.}
        \subsubsection{Электродвижущая сила (ЭДС).}
            \paragraph{Внутреннее сопротивление источника тока.}
        \subsubsection{Закон Ома для участка цепи.}
        \subsubsection{Закон Ома для полной электрической цепи.}
        \subsubsection{Источник напряжения и источник тока.}
        \subsubsection{Гальванометр.}
        \subsubsection{Правила Кирхгофа.}
            \paragraph{Падение напряжения на сопротивлении, емкости и вольтметре.}
            \paragraph{Первое правило Кирхгофа.}
            \paragraph{Второе правило Кирхгофа.}
            \paragraph{Закон сохранения заряда в узлах с конденсаторами.}
            \paragraph{Переходные процессы, зарядка и разрядка конденсатора через сопротивление, время переходного процесса (RC).}
        \subsubsection{Закон изменения энергии в электрических цепях.}
            \paragraph{Работа ЭДС.}
            \paragraph{Работа электрического тока.}
            \paragraph{Мощность электрического тока.}
            \paragraph{Закон Джоуля-Ленца.}
        \subsubsection{Полупроводники.}
            \paragraph{Полупроводниковый диод.}
            \paragraph{Цепи с полупроводниковыми диодами.}
        \subsubsection{Электрический ток в неметаллах.}
            \paragraph{Электрический ток в электролите.}
            \paragraph{Электролиз.}
            \paragraph{Электрический ток в вакууме.}
        \subsubsection{Электровакуумные приборы.}
    
\end{document}