\documentclass{article}

\usepackage[utf8x]{inputenc}
\usepackage[english,russian]{babel}
\usepackage{cmap}
\usepackage{commath}
\usepackage{amsmath}
\usepackage{amsfonts}
\usepackage{mathtools}
\usepackage{amssymb}
\usepackage{parskip}
\usepackage{titling}
\usepackage{color}
\usepackage{hyperref}
\usepackage{cancel}
\usepackage{enumerate}
\usepackage{multicol}
\usepackage{graphicx}
\usepackage{docmute}
\usepackage{titlesec}
\usepackage[font=small,labelfont=bf]{caption}
\usepackage[a4paper, left=2.5cm, right=1.5cm, top=2.5cm, bottom=2.5cm]{geometry}

\graphicspath{ {./images/} }
\setlength{\droptitle}{-3cm}
\hypersetup{ colorlinks=true, linktoc=all, linkcolor=blue }
\pagenumbering{arabic}

\setcounter{secnumdepth}{4}
\setcounter{tocdepth}{4}
\titleformat{\paragraph}
{\normalfont\normalsize\bfseries}{\theparagraph}{1em}{}
\titlespacing*{\paragraph}
{0pt}{3.25ex plus 1ex minus .2ex}{1.5ex plus .2ex}

\begin{document}

\section{Приложения.}
    \subsection{Константы.}

    \begin{center}
        \begin{tabular}{ l l }
            \hline
            Гравитационная постоянная & \(G = 6.67 \cdot 10^{-11} \frac{\textrm{Н} \cdot \textrm{м}^2}{\textrm{кг}^2}\) \\ 
            Электрическая постоянная & \(\varepsilon_0 = 8.85 \cdot 10^{-12} \frac{\textrm{Ф}}{\textrm{м}}\) \\  
            Масса электрона & \(m_e = 9.11 \cdot 10^{-31} \) кг \\
            Масса протона & \(m_p = 1.67 \cdot 10^{-27} \) кг \\
            Масса нейтрона & \(m_n = 1.675 \cdot 10^{-27} \) кг \\
            Элементарный заряд & \(e = 1.6 \cdot 10^{-19} \) Кл \\
            Коэффициент пропорциональности в законе Кулона & \(k = \frac{1}{4\pi\varepsilon_0} = 9 \cdot 10^{9} \frac{\textrm{Н} \cdot \textrm{м}^2}{\textrm{Кл}^2}\) \\
            \hline
        \end{tabular}
    \end{center}

    \subsection{Формулы.}
    
    \begin{center}
        \begin{tabular}{ l l }
            \hline
            \hline
            \multicolumn{2}{c}{Электростатика} \\
            \hline
            \hline
            Закон сохранения заряда & \(\Sigma q_i = const\) \\
            \hline
            Закон Кулона & \(F = k\frac{q_1 \cdot q_2}{r^2}\) \\
            \hline
            Напряженность электрического поля точечного заряда & \(E = k\frac{q}{r^2}\) \\
            \hline
            Электрическая сила & \( \vec{F} = q\vec{E} \) \\
            \hline
            Принцип суперпозиции электрических полей & \( \vec{E} = \vec{E_1} + \vec{E_2} + ... + \vec{E_n} \) \\
            \hline
            Потенциал электрического поля точечного заряда & \( \varphi = k\frac{q}{r} \) \\
            \hline
            Диэлектрическая проницаемость среды & \( \varepsilon = \frac{F}{F_0} \) \\
            \hline
            Напряжение & \( U = \varphi_1 - \varphi_2 \) \\
            \hline
            Работа электрического поля & \( A = qU \) \\
            \hline
            Связь напряженности и разности потенциалов & \( E = \frac{U}{d} \) \\
            \hline
            Электроемкость плоского конденсатора & \( C = \frac{q}{U} = \frac{\varepsilon S}{d} \) \\
            \hline
            Энергия конденсатора & \( W = \frac{CU^2}{2} \) \\
            \hline
            \hline
            \multicolumn{2}{c}{Электродинамика} \\
            \hline
            \hline
            Сила тока & \( I = \frac{dq}{dt} = envS \) \\
            \hline
            Сопротивление проводника & \( R = \frac{\rho l}{S} \) \\
            \hline
            Закон Ома для участка цепи & \( I = \frac{U}{R} \) \\
            \hline
            Закон Ома для полной цепи & \( I = \frac{\varepsilon}{R + r} \) \\
            \hline
            Законы последовательного соединения проводников & \(\begin{array}{l}
                I = const \\
                U = U_1 + U_2 + ... + U_n \\
                R = R_1 + R_2 + ... + R_n
            \end{array}\) \\
            \hline
            Законы последовательного соединения проводников & \(\begin{array}{l}
                I = I_1 + I_2 + ... + I_n \\
                U = const \\
                \frac{1}{R} = \frac{1}{R_1} + \frac{1}{R_2} + ... + \frac{1}{R_n}
            \end{array}\) \\
            \hline
            Закон Джоуля-Ленца & \( Q = I^2Rt = IUt \) \\
            \hline
            Мощность тока & \( Q = I^2R = IU \) \\
            \hline
            \hline 
        \end{tabular}
    \end{center}

\end{document}