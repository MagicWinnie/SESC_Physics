\documentclass{article}

\usepackage[utf8x]{inputenc}
\usepackage[english,russian]{babel}
\usepackage{cmap}
\usepackage{commath}
\usepackage{amsmath}
\usepackage{amsfonts}
\usepackage{mathtools}
\usepackage{amssymb}
\usepackage{parskip}
\usepackage{titling}
\usepackage{color}
\usepackage{hyperref}
\usepackage{cancel}
\usepackage{enumerate}
\usepackage{multicol}
\usepackage{graphicx}
\usepackage{docmute}
\usepackage{titlesec}
\usepackage[font=small,labelfont=bf]{caption}
\usepackage[a4paper, left=2.5cm, right=1.5cm, top=2.5cm, bottom=2.5cm]{geometry}

\graphicspath{ {./images/} }
\setlength{\droptitle}{-3cm}
\hypersetup{ colorlinks=true, linktoc=all, linkcolor=blue }
\pagenumbering{arabic}

\setcounter{secnumdepth}{4}
\setcounter{tocdepth}{4}
\titleformat{\paragraph}
{\normalfont\normalsize\bfseries}{\theparagraph}{1em}{}
\titlespacing*{\paragraph}
{0pt}{3.25ex plus 1ex minus .2ex}{1.5ex plus .2ex}

\begin{document}

\section{Магнетизм.}
    \subsection{Индукция магнитного поля.}
        \subsubsection{Векторная связь индукции магнитного поля с напряженностью электрического поля.}
        \subsubsection{Принцип суперпозиции векторов магнитной индукции.}
    \subsection{Напряженность магнитного поля.}
    \subsection{Магнитное поле проводника с током.}
    \subsection{Закон Био-Савара.}
    \subsection{Формула Стокса.}
    \subsection{Расчет индукции магнитного поля в простых системах.}
    \subsection{Поверхностный ток.}
    \subsection{Магнитное поле соленоида.}
    \subsection{Сила Лоренца.}
        \subsubsection{Движение заряда в однородном постоянном магнитном поле.}
        \subsubsection{Циклотронный резонанс.}
        \subsubsection{Масс-спектрометр.}
        \subsubsection{Эффект Холла.}
    \subsection{Движение заряженных частиц в скрещенных (электрическом и магнитном) полях.}
    \subsection{Сила Ампера.}
    \subsection{Магнитный момент в магнитном поле.}
    \subsection{Взаимодействие токов.}
    \subsection{Силовое воздействие магнитного поля на поверхность (давление).}
    \subsection{Электромагнитная индукция.}
        \subsubsection{Магнитный поток.}
        \subsubsection{ЭДС индукции.}
            \paragraph{В подвижных проводниках.}
            \paragraph{В переменном магнитном поле.}
            \paragraph{Закон электромагнитной индукции Фарадея.}
            \paragraph{Правило Ленца.}
            \paragraph{Электрические цепи с ЭДС индукции.}
        \subsubsection{Вихревое электрическое поле.}
    \subsection{Самоиндукция.}
        \subsubsection{Коэффициент самоиндукции (индуктивность).}
        \subsubsection{Индуктивность соленоида.}
    \subsection{Взаимная индукция.}
        \subsubsection{Коэффициент взаимной индукции.}
    \subsection{Энергия магнитного поля.}
        \subsubsection{Связь с индуктивностью системы.}
        \subsubsection{Плотность энергии магнитного поля.}
    \subsection{Сверхпроводники.}
        \subsubsection{Принцип сохранения магнитного потока.}
        \subsubsection{Сверхпроводящие контуры в электрических цепях.}
        \subsubsection{Сверхпроводники в магнитном поле.}
    \subsection{Ферромагнетики.}
        \subsubsection{Устройство и свойства магнита.}
        \subsubsection{Взаимодействие магнитов.}
        \subsubsection{Граничные условия для индукции и напряженности магнитного поля.}
        \subsubsection{Магнитопроводы.}
        \subsubsection{Трансформаторы, коэффициент трансформации.}
\section{Колебания.}
    \subsection{Свободные колебания.}
        \subsubsection{Амплитуда и фаза колебаний.}
        \subsubsection{Период колебаний.}
        \subsubsection{Частота колебаний. Собственная (круговая) частота колебаний.}
        \subsubsection{Эффект Доплера.}
        \subsubsection{Гармонические колебания.}
            \paragraph{Уравнение гармонических колебаний.}
        \subsubsection{Механические колебания.}
            \paragraph{Расчет собственной частоты колебаний через силы и моменты сил.}
            \paragraph{Расчет собственной частоты колебаний через энергию.}
            \paragraph{Свободные колебания математического маятника.}
            \paragraph{Свободные колебания пружинного (физического) маятника.}
        \subsubsection{Гармоническое движение.}
        \subsubsection{Гармонические электромагнитные колебания.}
        \subsubsection{Свободные электромагнитные колебания и колебательный контур.}
    \subsection{Вынужденные колебания.}
        \subsubsection{Резонанс.}
        \subsubsection{Вынужденные электромагнитные колебания и резонанс.}
        \subsubsection{Переменный ток.}
            \paragraph{Сопротивление, конденсатор и катушка индуктивности в цепи переменного тока.}
            \paragraph{Активное и реактивное сопротивления.}
            \paragraph{Импеданс.}
            \paragraph{Электрический резонанс в цепях переменного тока.}
            \paragraph{Колебания в цепях с трансформатором.}
            \paragraph{Мощность в цепях переменного тока.}
            \paragraph{Производство, передача и потребление электрической энергии.}
    \subsection{Затухающие колебания.}
        \subsubsection{Коэффициент затухания.}
        \subsubsection{Добротность.}
    \subsection{Звуковые и электромагнитные волны.}
        \subsubsection{Уравнение гармонической волны.}
\section{Оптика.}
    \subsection{Электромагнитное поле и волновая оптика.}
        \subsubsection{Свет как электромагнитная волна.}
        \subsubsection{Скорость электромагнитных волн, скорость света в вакууме и в среде.}
        \subsubsection{Показатель преломления и его связь со скоростью света.}
        \subsubsection{Дифракция света.}
            \paragraph{Дифракция от двух щелей.}
            \paragraph{Дифракционная решетка.}
    \subsection{Геометрическая оптика.}
        \subsubsection{Прямолинейное распространение света.}
        \subsubsection{Тени и полутени.}
        \subsubsection{Закон отражения света.}
            \paragraph{Построение изображений в плоском зеркале.}
        \subsubsection{Закон преломления света (закон Снеллиуса).}
            \paragraph{Плоскопараллельная пластинка.}
            \paragraph{Принцип Ферма.}
            \paragraph{Призма.}
        \subsubsection{Полное внутреннее отражение.}
            \paragraph{Световоды.}
        \subsubsection{Формула тонкой сферической линзы.}
        \subsubsection{Оптическая сила линзы.}
        \subsubsection{Построение изображений в линзах.}
        \subsubsection{Системы линз и зеркал.}
        \subsubsection{Глаз как оптическая система.}
    \subsection{Фотометрия.}
        \subsubsection{Освещенность.}
        \subsubsection{Световой поток.}
        \subsubsection{Сила света.}
    \subsection{Оптические приборы.}
        \subsubsection{Микроскоп.}
        \subsubsection{Телескоп.}
\section{Квантовая физика.}
    \subsection{Фотон.}
        \subsubsection{Энергия фотона.}
        \subsubsection{Импульс фотона.}
    \subsection{Фотоэффект.}
        \subsubsection{Уравнение А.Эйнштейна для фотоэффекта.}
\section{Физика атома, атомного ядра и элементарных частиц.}
    \subsection{Нуклонная модель ядра.}
        \subsubsection{Заряд ядра.}
        \subsubsection{Массовое число ядра.}
        \subsubsection{Изотопы.}
    \subsection{Энергия связи ядра (нуклонов в ядре).}
    \subsection{Дефект массы ядра.}
    \subsection{Ядерные реакции.}
        \subsubsection{Деление и синтез ядер.}
        \subsubsection{Период полураспада.}
    \subsection{Радиоактивность.}
        \subsubsection{Альфа-распад.}
        \subsubsection{Бетта-распад.}
        \subsubsection{Гамма-излучение.}
    \subsection{Закон радиоактивного распада.}
    \subsection{Цепная реакция деления ядер.}
\section{Основы специальной теории относительности.}
    \subsection{Постулаты специальной теории относительности Эйнштейна.}
        \subsubsection{Инвариантность скорости света.}
        \subsubsection{Принцип относительности Эйнштейна.}
    \subsection{Пространство и время в специальной теории относительности.}
        \subsubsection{Относительность одновременности.}
        \subsubsection{Преобразование Лоренца.}
        \subsubsection{Преобразование длины.}
        \subsubsection{Преобразование скоростей.}
    \subsection{Энергия и импульс.}
        \subsubsection{Связь массы и энергии.}
        \subsubsection{Энергия покоя.}
        \subsubsection{Полная энергия.}
        \subsubsection{Релятивистский импульс.}
        \subsubsection{Связь полной энергии с импульсом и массой тела.}

    
\end{document}